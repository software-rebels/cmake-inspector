%! Author = mehran
%! Date = 7/28/20

\section{Study Design}
\label{sec:design}

We analyzed all the open source projects that used CMake as their build management system in this paper.
We carefully vet all the projects and used multiple filters to remove toy ones.
In this section, first we talk about the filtration process.
Then, we move on to data extraction, and finally, we explain the analysis procedure.

\subsection{Subject Systems/Communities}\label{subsec:subject-systems/communities}

This section should present a strong argument as to why the subject systems/communities were selected.
For studies that, due to scarcity of available data, have a small number of subjects, it is often useful to present the set of criteria that must be met by the study subjects.
This can then be used to take a broader set of candidate systems and whittle them down to the set that satisfied the criteria.
This can help reviewers to understand why a larger scale analysis was not possible.

\subsection{Data Extraction}

After selecting the subject systems, it is necessary to extract relevant data.
This process is typically broken down into a few steps within the overview figure.
Use indices (e.g., ``(DE1)'') on the steps in the figure and match those indexes in the text below that describes the details of each step.
Use active prose to describe the steps.

\smallsection{(DE1) Download Raw Data}
An example of a data extraction step.
In the first paragraph, explain why the step is necessary.

In one or two more paragraphs, explain the relevant, high-level details about how the step is implemented.

\smallsection{(DE2) Establish Links Between Elements}
Another example of a data extraction step.
In the first paragraph, explain why the step is necessary.

In one or two more paragraphs, explain the relevant, high-level details about how the step is implemented.

\subsection{Data Filtering}

Software data is often noisy and imprecise.
In this section, explain the steps that were taken to mitigate noise and remove irrelevant data.
Again, use indices (e.g., ``(DF1)'') to label the steps in the figure and identify the text below.
Use active prose to describe the steps.

\smallsection{(DF1) Filter Inactive Projects}
First, explain what the source of noise is and why it would be problematic for the study.
For example, ``GitHub contains a large number of toy projects that have not yet reached maturity. Since active software projects face unique challenges, including inactive projects in our sample of studied systems may lead us to incorrect conclusions.''

In this next paragraph, explain how the filter was implemented and the impact that it has had on the data set (e.g., how many examples/projects/rows survive the filter).

\subsection{Data Analysis}

Now that the data has been extracted and cleaned up, it is ready for analysis.
In this section, use the same procedure to explain the steps in the data analysis pipeline.

\smallsection{(DA1) Conduct Statistical Tests}
Explain the goal of the analysis.
Explain the analysis instruments that were selected and why alternatives were not selected.